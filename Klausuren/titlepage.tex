
\renewcommand{\arraystretch}{1.7}
\hspace{-2em}
\begin{tabularx}{\textwidth}{|X|X|X|X|}
	\hline
	\multicolumn{2}{|X|}{%
	\multirow{2}{*}{%
		\includegraphics[width=.43\textwidth]{THD_Logo.pdf}
	}
	}
	& \multicolumn{2}{l|}{Matrikelnummer:}\\
	\cline{3-4}
	\multicolumn{2}{|p{.5\textwidth}|}{}
	& Platzziffer: & Punkte:\quad\quad Note:\quad\\
	\hline
	\multicolumn{4}{|c|}{\textbf{\faculty}}\\
	\hline
	\textbf{Kurs:} & \course & \textbf{Semester:} & \semester \\
	\hline
	\textbf{Studiengang:} & \programme & \textbf{Dauer/Anteil:} & \duration \\
	\hline
	\textbf{Prüfer:} & \theauthor & \textbf{Prüfungsdatum} & \thedate \\
	\hline
	\textbf{Hilfsmittel:} & \supplies & \textbf{Uhrzeit} & \thetime \\
	\hline
	\textbf{Prüfungsart:} & \examtype & \textbf{Anzahl d. Blätter} & \numpages \\
	\hline
\end{tabularx}

\vfill


\textbf{Bevor Sie beginnen:}
\begin{itemize}
	\item Kontrollieren Sie ob Ihr Angabenblatt vollständig ist.
	\item Schreiben Sie Ihre Platzziffer auf jede Seite.
	\item Lesen Sie sich alle Aufgaben durch und verschaffen Sie sich einen Überblick über den Umfang der Aufgaben.
	\item Stellen Sie sicher, dass Ihre Antworten vollständig und lesbar sind.
	\item Schreiben Sie nicht außerhalb des markierten Seitenrandes.
\end{itemize}

\textbf{Viel Erfolg!}

\vfill

\begin{center}
	\gradetable[h]{}
\end{center}

\vfill

\clearpage
